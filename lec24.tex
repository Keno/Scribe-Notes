%% LyX 2.0.4 created this file.  For more info, see http://www.lyx.org/.
%% Do not edit unless you really know what you are doing.
\documentclass[11pt]{article}
\usepackage[latin9]{inputenc}
\usepackage{amsmath}
\usepackage{amssymb}

\makeatletter
%%%%%%%%%%%%%%%%%%%%%%%%%%%%%% Textclass specific LaTeX commands.
\newenvironment{lyxcode}
{\par\begin{list}{}{
\setlength{\rightmargin}{\leftmargin}
\setlength{\listparindent}{0pt}% needed for AMS classes
\raggedright
\setlength{\itemsep}{0pt}
\setlength{\parsep}{0pt}
\normalfont\ttfamily}%
 \item[]}
{\end{list}}

%%%%%%%%%%%%%%%%%%%%%%%%%%%%%% User specified LaTeX commands.

\usepackage{amsthm}

\DeclareMathOperator*{\E}{\mathbb{E}}
\let\Pr\relax
\DeclareMathOperator*{\Pr}{\mathbb{P}}

\newcommand{\eps}{\varepsilon}
\newcommand{\inprod}[1]{\left\langle #1 \right\rangle}
\newcommand{\R}{\mathbb{R}}

\newcommand{\handout}[5]{
  \noindent
  \begin{center}
  \framebox{
    \vbox{
      \hbox to 5.78in { {\bf CS 224: Advanced Algorithms } \hfill #2 }
      \vspace{4mm}
      \hbox to 5.78in { {\Large \hfill #5  \hfill} }
      \vspace{2mm}
      \hbox to 5.78in { {\em #3 \hfill #4} }
    }
  }
  \end{center}
  \vspace*{4mm}
}

\newcommand{\lecture}[4]{\handout{#1}{#2}{#3}{Scribe: #4}{Lecture #1}}

\newtheorem{theorem}{Theorem}
\newtheorem{corollary}[theorem]{Corollary}
\newtheorem{lemma}[theorem]{Lemma}
\newtheorem{observation}[theorem]{Observation}
\newtheorem{proposition}[theorem]{Proposition}
\newtheorem{definition}[theorem]{Definition}
\newtheorem{claim}[theorem]{Claim}
\newtheorem{fact}[theorem]{Fact}
\newtheorem{assumption}[theorem]{Assumption}

% 1-inch margins, from fullpage.sty by H.Partl, Version 2, Dec. 15, 1988.
\topmargin 0pt
\advance \topmargin by -\headheight
\advance \topmargin by -\headsep
\textheight 8.9in
\oddsidemargin 0pt
\evensidemargin \oddsidemargin
\marginparwidth 0.5in
\textwidth 6.5in

\parindent 0in
\parskip 1.5ex

\makeatother

\begin{document}
\lecture{24 -- Nov. 20, 2014}{Fall 2014}{Prof.\ Jelani Nelson}{Xiaoyu
He}


\section*{Overview}

Today we will move to a new topic: another way to deal with NP-hard
problems. We have already seen approximation algorithms for NP-Hard
algorithms. This lecture we will bite the bullet and solve NP-hard
problems in the ``fastest'' possible exponential time. Here are
the techniques we will cover and the problems we will apply them to:
\begin{enumerate}
\item Exponential Divide and Conquer $\implies$ Travelling Salesman.
\item Pruned brute force $\implies$ 3-SAT.
\item Randomization $\implies$ 3-SAT.
\item Inclusion-Exclusion/Fast M�bius Transform $\implies$ k-coloring.
\end{enumerate}
Our goal is to find the best time/space complexity for exponential
time solutions to these problems. Sometimes we can reduce to polynomial
space even when the natural implementation of some algorithm requires
exponential space. Throughout we drop polynomial factors: the notation
$O^{*}(f(n))$ means $O(\mbox{poly}(n)f(n))$, i.e. up to polynomial
factors.

For each problem, we first describe naive brute-force solutions and
then show how to improve the exponential part of the time or space
complexity.


\section{Exponential Divide and Conquer $\implies$ Travelling Salesman}

Here is our formulation of travelling salesman. The input is
\begin{itemize}
\item Set $P$ of points.
\item Metric distance function $d:P\times P\rightarrow\R_{\geq0}$ (i.e.
makes $P$ into a finite metric space). 
\item Two identified points $s,t\in P$.
\end{itemize}
Goal: Visit every vertex starting at $s$, ending at $t$, minimizing
total tour length. 

Remark: it is an open problem to find an algorithm for travelling
salesman in $O^{*}(2^{n})$ time with poly$(n)$ space. Can get one
or the other combining the following algorithms.


\subsection{Brute Force}

If we try all $(n-2)!$ paths, each path taking $n$ time to check,
runtime is $O^{*}(n!)$, and space is poly$(n)$.


\subsection{Dynamic Programming}

With dynamic programming we can improve to $O^{*}(2^{n})$ time and
at the cost of also using $O^{*}(2^{n})$ space. 

Define $f(x,S)$ to be the length of the shortest $x-t$ path visiting
every node in $S$ exactly once. Dynamically solve for all $f(x,S)$
for all pairs $x,S$, using
\[
f(x,S)=\min_{y\in S}\{\mbox{dist}(x,y)+f(y,S-\{x\})\}.
\]
The minimal path is $f(s,P-\{s,t\})$. Dynamic programming needs to
store $O(n\cdot2^{n})$ states, making $O(n)$ steps from each step,
so the time and space are both $O^{*}(2^{n})$.


\subsection{Divide and Conquer}

If the exponential space is prohibitive, we can get $O((4+\eps)^{n})$
time, poly$(n)$ space, for any $\eps>0$.

Let $OPT(U,s,t)$ be the length of the cheapest s-t tour touching
each point in $U\subseteq P$ once. Divide and Conquer by splitting
the path into two halves in all possible ways:

\[
OPT(U,s,t)=\min_{S,T,m}OPT(S,s,m)+OPT(T,m,t)
\]


We take the minimum over all $O^{*}(\binom{|U|}{|U|/2})$ ways of
partitioning $U$ into subsets $S,T$ and an intermediate point $m\neq s,t$
satisfying $|S|=\lfloor\frac{|U|}{2}\rfloor+1$, $T\cup S=U$, $T\cap S=m$.
Compute $OPT(P,s,t)$ recursively; the base case is $|U|\le2$. Space
is poly$(n)$. Time satisfies

\[
T(n)=(n-2)\binom{n-2}{\lceil\frac{n-2}{2}\rceil}\cdot2\cdot T(n/2),
\]
which solves to $T(n)=O(4^{n}\cdot n^{C\log n})=O^{*}((4+\varepsilon)^{n})$
for all $\varepsilon>0$, where $T(n)$ is the time for $n$ nodes.


\section{Pruned Brute Force and Randomization $\implies$ 3-SAT}

The input for 3-SAT involves:
\begin{itemize}
\item Logical expression $\phi=C_{1}\wedge C_{2}\wedge\ldots\wedge C_{m}$.
\item Each clause $C_{j}$ is a $\vee$ of three literals: $x_{i_{1}}\vee x_{i_{2}}\vee x_{i_{3}}$
(possibly negated)
\item Variables $x_{i}$, $i=1,\ldots,n$.
\end{itemize}
Goal: Decide whether there exists an assignment $x=a$ of truth values
which makes $\phi$ evaluate to true.

Remark: There are at most $O(n^{3})$ distinct possible clauses $C_{i}$
so we may assume $m=O(\mbox{poly}(n))$.

The current records for 3-SAT are randomized $O^{*}(1.30704^{n})$
\cite{Hertli} and deterministic $O^{*}(1.3303^{n})$ \cite{MSM}.


\subsection{Brute Force}

Try all $2^{n}$ assignments: time $O^{*}(2^{n})$, space poly$(n)$.


\subsection{Pruned Brute Force}

The following example of pruned brute force is from \cite[Section 4]{Erickson} (further speedups along these lines are also discussed there).

We introduce the following conditional notation: $\phi|x_{1}\bar{x_{2}}$
is the new expression obtained from $\phi$ by setting $x_{1}$ true,
$x_{2}$ false, and simplifying.

Let $C_{1}=x\vee y\vee z$ (possibly with negations). Try all three
possibilities recur: 
\begin{lyxcode}
def~SAT($\phi$):~
\begin{lyxcode}
if~SAT($\phi|x$)~return~True~

else~if~SAT($\phi|y$)~return~True

else~if~SAT($\phi|z$)~return~True

else~return~False
\end{lyxcode}
\end{lyxcode}
The runtime is $O^{*}(3^{n})$, since each recursive step reduces
the number of variables remaining by at least one. Thus if $T(n)$
is the maximum amount of time to solve an instance $\phi$ with $n$
variables, we have $ $$T(n)\leq3T(n-1)+O(1)$.

This is worse than regular brute force. However, we can offer the
following improvement: if we reach the second if, then the first one
failed and we may assume $\bar{x}$! Thus:
\begin{lyxcode}
def~SAT($\phi$):~
\begin{lyxcode}
if~SAT($\phi|x$)~return~True~

else~if~SAT($\phi|\bar{x}y$)~return~True~

else~if~SAT($\phi|\bar{x}\bar{y}z$)~return~True~

else~return~false
\end{lyxcode}
\end{lyxcode}
Our improved runtime satisfies $T(n)\leq T(n-1)+T(n-2)+T(n-3)+O(1)$,
which solves to $O^{*}(1.8393^{n})$, the constant coming from the
largest real root of $x^{3}-x^{2}-x-1=0$. This pruning algorithm
can be further improved, but we will show something even better.


\subsection{Another Brute Force}

Suppose $a^{*}$ satisfies $\phi$. WLOG, we know an assignment $a$
s.t. $dist(a,a^{*})\leq n/2$ (run the following algorithm twice,
one assuming $a=0$ all zeroes, one assuming $a=1$ all ones), where
the distance here is the Hamming distance.

Start with $a$. While there exists an unsatisfied $C$, try all three
of the variables in $C$ and flip their assignments in $a$, recurse
for each of them. Limit the recursion depth to $n/2$ iterations.
Given our assumption that $a^{*}$ is within $n/2$ Hamming distance
away from $a$, this algorithm must find $a^{*}$. The runtime is
$O^{*}(3^{n/2})=O^{*}(1.7321^{n})$. This is already better than the
previous solution, and we will see how to further improve it via randomization.


\subsection{Randomization }

In this section we present Sch\"{o}ning's randomized version \cite{Schoning} of the algorithm
in the previous section, achieving runtime $O^{*}((4/3)^{n})$, and
space poly$(n)$.
\begin{lyxcode}
Algorithm~S:~
\begin{lyxcode}
Pick~$a$~uniformly~at~random.~

Repeat~at~most~$t$~times:
\begin{lyxcode}
(i)~Let~$C$~be~the~first~clause~not~satisfied~by~$a$.~

(ii)~Pick~a~random~variable~in~$C$~and~flip~its~assignment~in~$a$.
\end{lyxcode}
\end{lyxcode}
\end{lyxcode}
Sch�ning's algorithm is to repeat $S$ until we succeed. The parameter
$t$ we will tune later.

Remark: Similar algorithm for 2-SAT \cite{Papadimitriou} needs to
run for $O(n^{2})$ steps to find SAT assignment with probability
$\geq2/3$.


\subsubsection{Analysis of Sch�ning's algorithm}

Recall the definition of a Markov chain: a directed graph, where every
edge $e$ has a assigned probability $p(e)$, such that for all vertices
$v$,

\[
\sum_{e\mbox{ leaving}v}p_{e}=1.
\]
Let $G$ be a path on $n+1$ vertices, with edges in both directions
between adjacent vertices. Vertex $i$ represents all assignments
$a$ satisfying $dist(a,a^{*})=i$ (fix a solution $a^{*}$ if there
are multiple). At every step in Algorithm S, we are moving left or
right on the graph one step, i.e. this is a random walk on ${0,\ldots,n}$.

At an intermediate vertex $0<d<n$, there are $3$ variables in the
clause $C$ we pick, at least one of which $a$ disagrees with $a^{*}$
on, so the probability of going left towards $0$ is at least $1/3$.

Thus, if we calculate the expected time of hitting $0$ by going left
with probability $1/3$ at every point on an infinite ray $0,1,\ldots$
this is an upper bound for the amount of time to get to $0$ in algorithm
S. We simplify even more and only compute the probability of hitting
$0$ at exactly the $t$-th step.

Starting off at distance $k$ from $0$, what is the probability that
we land at $0$ at exactly time $t=3k$? 

We will show that this probability is at least $(3/4)^{n}$:

\[
P[k\mbox{ steps to the right},2k\mbox{ to the left}]\geq\binom{3k}{k}\Big(\frac{1}{3}\Big)^{2k}\cdot\Big(\frac{2}{3}\Big)^{k}.
\]


By Stirling's formula, and some calculation, we get that the probability
above is

\[
\Theta\Big(\frac{1}{\sqrt{k}}\cdot\frac{3^{3k}}{2^{2k}}\cdot\Big(\frac{1}{3}\Big)^{2k}\cdot\Big(\frac{2}{3}\Big)^{k}\Big)=\Theta\Big(\frac{1}{\sqrt{k}}\cdot\frac{1}{2^{k}}\Big).
\]


Thus averaging over all possibilities for $k$, weighted by their
probabilities:
\begin{eqnarray*}
P[\mbox{success}] & \geq & \sum_{k=0}^{n}P[\mbox{dist}(a,a^{*})=k]\cdot\Theta(\frac{1}{\sqrt{k}}\cdot\frac{1}{2^{k}})\\
 & \geq & \frac{1}{\sqrt{n}}\sum_{k=1}^{n}\binom{n}{k}\frac{1}{2^{n+k}}\\
 & = & \frac{1}{\sqrt{n}}\frac{1}{2^{n}}\Big(1+\frac{1}{2}\Big)^{n}\\
 & = & \frac{1}{\sqrt{n}}\frac{3^{n}}{4^{n}},
\end{eqnarray*}
which is the order $P=n^{-1/2}(3/4)^{n}$ desired. Thus if we run
algorithm $S$ at least $T = \ln(1/\delta)/P$ times with $t=3n$, the probability of not finding a satisfying assignment any any of the iterations is at most $(1-P)^T \le e^{-T/P} \le \delta$. Thus for constant success probability, the running time is $O^{*}((4/3)^{n})$.


\section{Inclusion-Exclusion $\implies$ $k$-coloring}

Given an undirected graph $G=(V,E)$, $|V|=n$, $|E|=m$, a $k$-coloring
of $G$ is a function $c:V\rightarrow\{1,\ldots,k\}$ such that $(u,v)\in E\implies c(u)\neq c(v)$,
i.e. no monochromatic edges.

Goal: Decide if a $k$-coloring exists. Equivalently, decide if $V$
can be partitioned into $k$ independent sets (sets of vertices with
no edges between them).


\subsection{Brute Force}

If we simply try all possible colorings, this takes $O^{*}(k^{n})$
time, poly$(n)$ space. 


\subsection{Dynamic Programming}

We compute with dynamic programming $F(t,S)=$ indicator of whether
or not $S\subseteq V$ can be partitioned into $t$ independent sets;
we want $F(k,V)$.

To compute $F(t,S)$, we use the recurrence:

\[
F(t,S)=\bigvee_{T\subset S}(T\mbox{ independent}\wedge F(t-1,S-T))
\]


The runtime is at most

\[
\sum_{S\subseteq V}2^{|S|}=O^{*}(3^{n})
\]
by the binomial theorem. The space is $O^{*}(2^{n})$ since we will
be storing $\mbox{poly}(n)$ data for each subset $S\subseteq V$.


\subsection{Inclusion-Exclusion}

Here we follow the presentation of \cite{Husfeldt} (see that survey for more history and references).

Recall the statement of the Principle of Inclusion/Exclusion:

\begin{theorem} (Inclusion/Exclusion) For any sets $R\subseteq T$,

\[
\sum_{R\subseteq S\subseteq T}(-1)^{|T-S|}=[R=T],
\]
i.e. the sum is zero unless $R=T$. \end{theorem}

\begin{proof} If $R=T$ then the sum has just one term $1$. Otherwise,
pick some $t\in T-R$, so that $S\mapsto S\oplus\{t\}$ (i.e. put
in $t$ if it is not there, remove it if it is) is a bijection between
the terms with $|T-S|$ even and $|T-S|$ odd. Thus the whole sum
cancels. \end{proof}

Define $f(S)$ to be $1$ iff $S$ is a nonempty independent set,
and define the dual function $\hat{f}(S)=\sum_{R\subseteq S}f(R)$.

\begin{claim}$G$ is $k$-colorable iff 
\begin{equation}
\sum_{S\subseteq V}(-1)^{n-|S|}\hat{f}(S){}^{k}>0.\label{eq:colorable}
\end{equation}
\end{claim}

\begin{proof} Each $\hat{f}(S)^{k}$ expands as a sum
\[
\hat{f}(S)^{k}=\sum_{R_{i}\subseteq S}f(R_{1})f(R_{2})\cdots f(R_{k}),
\]
over all choices of $k$ subsets $R_{i}$ of $S$. These vanish unless
all of the $R_{i}$ are independent. For any fixed choice of $k$
independent sets $R_{1},R_{2},\ldots,R_{k}\subseteq V$, the coefficient
of $f(R_{1})\cdots f(R_{k})$ in \eqref{eq:colorable} is just
\[
\sum_{R\subseteq S\subseteq V}(-1)^{|V-S|},
\]
where $R=\bigcup R_{i}$. By Inclusion/Exclusion, the only terms that
remain are those in which $R=V$, and these are positive. If there
is such a term, $V$ can be written as a union of $k$ independent
sets, as desired. Although the $R_{i}$ themselves need not be disjoint,
we can remove elements from them until they are disjoint; they will
remain a partition of $V$ by independent sets. \end{proof}

Thus we have reduced the problem to: efficiently compute $\hat{f}(S)$
for all $S$. We can achieve either $O^{*}(3^{n})$ time and poly$(n)$
space or $O^{*}(2^{n})$ time and space.

\begin{thebibliography}{1}

\bibitem{Erickson} Jeff Erickson. Algorithms. Download at \texttt{http://web.engr.illinois.edu/\textasciitilde{}jeffe/teaching/algorithms/}.

\bibitem{Hertli} Timon Hertli. ``3-SAT Faster and Simpler---Unique-SAT
Bounds for PPSZ Hold in General.'' {\em SIAM Journal on Computing}, 43(2), pp. 718--729, 2014.

\bibitem{Husfeldt} Thore Husfeldt. Invitation to Algorithmic Uses of Inclusion-Exclusion. {\em ICALP}, pp. 42--59, 2011.

\bibitem{MSM} Kazuhisa Makino, Suguru Tamaki, and Masaki Yamamoto.
``Derandomizing the HSSW Algorithm for 3-SAT.'' {\em Algorithmica}, 67(2), pp. 112-1124, 2013.

\bibitem{Papadimitriou}Christos H. Papadimitriou. ``On selecting
a satisfying truth assignment.'' In {\em Proceedings of the 32nd Annual IEEE Symposium on Foundations of Computer Science},
pp. 163-169, 1991.

\bibitem{Schoning}Uwe Sch\"{o}ning. ``A probabilistic algorithm
for k-SAT based on limited local search and restart.'' {\em Algorithmica},
32, no. 4 (2002): 615-623.\end{thebibliography}

\end{document}
