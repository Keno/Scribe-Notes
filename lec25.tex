\documentclass[11pt]{article}
\usepackage{amsmath,amssymb,amsthm,hyperref}

\DeclareMathOperator*{\E}{\mathbb{E}}
\let\Pr\relax
\DeclareMathOperator*{\Pr}{\mathbb{P}}

\newcommand{\eps}{\varepsilon}
\newcommand{\inprod}[1]{\left\langle #1 \right\rangle}
\newcommand{\R}{\mathbb{R}}
\renewcommand{\P}{\mathbb{P}}
\newcommand\dif{\mathop{}\!\mathrm{d}}

\newcommand{\handout}[5]{
  \noindent
  \begin{center}
  \framebox{
    \vbox{
      \hbox to 5.78in { {\bf CS 224: Advanced Algorithms } \hfill #2 }
      \vspace{4mm}
      \hbox to 5.78in { {\Large \hfill #5  \hfill} }
      \vspace{2mm}
      \hbox to 5.78in { {\em #3 \hfill #4} }
    }
  }
  \end{center}
  \vspace*{4mm}
}

\newcommand{\lecture}[4]{\handout{#1}{#2}{#3}{Scribe: #4}{Lecture #1}}

\newtheorem{theorem}{Theorem}
\newtheorem{corollary}[theorem]{Corollary}
\newtheorem{lemma}[theorem]{Lemma}
\newtheorem{observation}[theorem]{Observation}
\newtheorem{proposition}[theorem]{Proposition}
\newtheorem{definition}[theorem]{Definition}
\newtheorem{claim}[theorem]{Claim}
\newtheorem{fact}[theorem]{Fact}
\newtheorem{assumption}[theorem]{Assumption}
\newtheorem{example}[theorem]{Example}

% 1-inch margins, from fullpage.sty by H.Partl, Version 2, Dec. 15, 1988.
\topmargin 0pt
\advance \topmargin by -\headheight
\advance \topmargin by -\headsep
\textheight 8.9in
\oddsidemargin 0pt
\evensidemargin \oddsidemargin
\marginparwidth 0.5in
\textwidth 6.5in

\parindent 0in
\parskip 1.5ex

\newcommand{\Oh}{\mathcal{O}}
\newcommand{\Ohstar}{\Oh^\ast}
\newcommand{\ohstar}{\Oh^\ast}
\newcommand{\poly}{\text{poly}}
\newcommand{\sets}{\textbf{sets }}
\newcommand{\query}{\text{query}}
\newcommand{\support}{\text{support}}
\newcommand{\Var}{\mathrm{Var}}

\begin{document}

\lecture{Lecture 25 --- November 25, 2014}{Fall 2014}{Prof.\ Jelani Nelson}{Keno Fischer}

\section{Today}

\begin{itemize}
    \item Finish faster exponential time algorithms (Inclusion-Exclusion/Zeta Transform, M\"obius inversion)
    \item Streaming/Sketching
\end{itemize}

\section{Last time}

\begin{fact}[Inclusion-Exclusion]
\[
\forall R \subset T \qquad \sum_{R \subset S \subset T} (-1)^{|T\backslash S|} = [ R = T ]
\]
\end{fact}

\begin{definition}[Zeta transform]

Given a function 
\[
f: \sets \to \text{some ring}
\]
let $\hat{f}$ be the zeta transform given by
\[
\hat{f}(S) = \sum_{R \subset S} f(R)
\]

\end{definition}

\begin{claim}[$k$-coloarbility]\label{k-color}
Define
\[
f(S) = [ \text{S is a non-empty independent set } ]
\]
Then, a graph $G$ is k-colorable iff $\sum_{S\in V} (-1)^{V\backslash S} (\hat{f}(S))^k > 0$
\end{claim}

Will see this proof as special case of something more general:
\begin{itemize}
\item $\Ohstar(3^n)$ time, $\poly(n)$ space ~\cite{BjorklundH06}
\item $\Ohstar(2^n)$ time/space ~\cite{koivisto20062}

\item The first of the 2 papers also gets $\Ohstar(2.2461^n)$ time and $\poly(n)$ space

\underline{OPEN:} $\Ohstar(2^n)$ time, $\poly(n)$ space.

\end{itemize}

\section{Zeta transforms, M\"obius inversion}

\subsection{More generally}

\begin{definition}{Zeta transform, M\"obius inversion}

Given 
\[f: \sets \to \text{some ring}\]
let the \textit{Zeta transform} of $\hat{f}$ of $f$ be given by
\[ \hat{f}(S) = \sum_{R \subset S} f(R) \]
and define the M\"obius inversion $\tilde{f}$ of $f$ to be
\[ \tilde{f}(S) = \sum_{R \subset S} (-1)^{|S \backslash R|} f(R) \]

\end{definition}

\begin{claim}[Zeta Inverse]
\label{zetam}
\[
\hat{\tilde{f}} = \tilde{\hat{f}} = f
\]
\end{claim}
\begin{proof}
We will show $\hat{\tilde{f}} = f$ (Showing $\tilde{\hat{f}}$ is similar)

\[ \tilde{f}(S) = \sum_{R\subset S} (-1)^{|S\backslash R|} f(R) \]
so
\[
\hat{\tilde{S}} = \sum_{S \subset T} \sum_{R \subset S} (-1)^{|S \backslash R|} f(R)
= \sum_{R \subset S \subset T} (-1)^{|S \backslash R|} f(R)
\]\[
\sum_{R \subset T} f(R) \sum_S [ R \subset S \subset T ] (-1)^{S}
\]
equals f(R) by inclusion exclusion.
\end{proof}

\subsection{Proving \autoref{k-color} (k-colorability)}

Define 
\[
g(S) = \text{\# of ways to write S as a union of $k$ non-empty independent sets}
\]
\begin{fact}
$g(V) > 0 \iff \text{G is k-colorable}$
\end{fact}

\begin{claim}
$g(V) = \tilde{\hat{g}}(V) = \overset{\sim}{\hat{f}^k}(V)$ (where f as in \autoref{k-color})
\begin{proof}

We will show that $\hat{g}(S) = (\hat{f}(S))^k$.
Under the definition of the zeta transform $\hat{g}(S)$ counts the number ways to construct $k$-independent subsets of $S$ such that there union is contained in $S$ (g counts the same for the union being equal to $S$).
Note that any union of $k$ non-empty independent subsets of $S$ satisfies, this definition, so $\hat{g}(S)$ is given by the number of independent sets raised to the $k$-th power, which is precisely the right hand side.

\end{proof}
\end{claim}

\begin{claim}
Can compute all $\hat{f}(S)$ (or $\tilde{f}(S)$) values sequentially in time $\Ohstar(3^n)$, $\poly(n)$ space
\end{claim}
\begin{proof}

\[ \hat{f}(S) = \sum_{R \subset S} f(R) \]

We want to compute $\hat{f}(S)$ for all $S$. Time up to star is $\sum_{R \subset S} 2^{|R|} = \sum_{i=1}^n {n \choose i} 2^i \leq 3^n$

\end{proof}

What about a faster algorithm?

\subsection{Yates' algorithm}

We can solve this problem in $\Ohstar(2^n)$ time/space using Yates' algorithm, which is a dynamic programming approach.


We want to compute $\hat{f}(S) \; \forall S \subset \{1, \ldots, n\}$. 

Define
\[
g(i,S) = \sum_{R \subset S} [S(i) = R(i)] \cdot f(R)
\]
\[
S(i) = S \cap \{i+1,\ldots,n\}
\]

Note: $\hat{f}(S) = g(n,S)$

\begin{claim}
We can compute $g(i,S)$ using the recurrence

\[
g(i,S) = \begin{cases}
f(S) &\mbox{if } i = 0 \\
g(i-1,S) + [i \in S] \cdot g(i-1,S\backslash \{i\}) &\mbox{if } i > 0
\end{cases}
\]

hence we can compute $g(n,S)$ in space $\Ohstar(2^n)$ and time proportional to space as every computation if a constant number of recursive calls plus possibly a constant number of arithmetic operations.

\begin{proof}
We can write
\[
g(i,S) =
    \sum_{\substack{
        R\subset S \\
        i \in R}}
            [S(i) = R(i)] \cdot f(R)
  + \sum_{\substack{
        R \subset S \\
        i \notin R}}
            [S(i) = R(i)] \cdot f(R)
\]

We have two cases:
\begin{itemize}
\item[$i \notin S$:]
Since $i \notin S$ and $R \subset S$, $i \notin R$, so the first sum is zero. Now, since $i \notin S$, $S(i) = S(i-1)$ and similarly $R(i) = R(i-1)$, so the second sum is given by
\[ \sum_{R \subset S} [ S(i-1) = R(i-1) ] \cdot f(R) = g(i-1,S) \]

\item[$i \in S$]:
Since $i$ is in both $R$ and $S$, $S(i) = R(i)$ iff $S(i-1) = R(i-1)$.

So the first sum equals:
\[ \sum_{R\subset S} [ S(i-1) = R(i-1) ] \cdot f(R) = g(i-1,S) \]
In the second sum we want $R$ which agrees with $S$ from $i$ onward, so it agrees with $S\backslash \{i\}$ from i onward:

\[
\sum_{R \subset S} [(S \backslash \{i \})(i-1) = R(i-1)] \cdot f(R) = g(i-1,S\backslash\{i\})
\]
\end{itemize}

\end{proof}
\end{claim}

We can solve other problems using this technique

\begin{itemize}
\item min steiner tree
\item find all k-colorable induced subgraphs
\item For more applications, see Husfeldt, ``Invitation to algorithmic uses of inclusion–exclusion'' ~\cite{2011arXiv1105.2942H}
\end{itemize}

To get the actual $k$-coloring (rather than just deciding whether one exists), there is a reduction given in ~\cite{BjorklundH06} that finds the $k$-coloring in $\Ohstar(2^n)$ time and space.

Computing the M\"obius inversion is left as an exercise.

\section*{Streaming and Sketching}
\begin{center}
\textbf{Final topic of class}
\end{center}

For more in depth exposure take ``Alogrithms for big data''.

\begin{itemize}
\item[\underline{First}] Streaming (small space data structures).
\item[\underline{Model}] We have some high dimensional vector $x \in \R^n$ (or high dimensional matrix $X \in R^{n\times n}$). We want to support updates to $x$ and queries about $x$.
\end{itemize}

\par
Will look at one kind of update: \underline{turnstile model}~\cite{muthukrishnan2005data}:

\begin{itemize}
\item Each update (i,v) causes $x_i \leftarrow x_i + v$
\item x starts off as $\vec{0} \in \R^n$
\item \underline{example queries}: \begin{itemize}
    \item What is $x_i$ ?
    \item What is $|\support(x)|$
\end{itemize}
\end{itemize}

There's always the solution of either remembering $x$ or the whole stream (so either $\approx n$ or $m$ space). 

(One) Goal: come up with algorithm that uses much less space than the trivial solution.

\begin{example}
$\query(x) = |\support(x)| = F_0$

All updates $(i,v)$ have $v=1$. Can use exactly $n$ bits of space (as a bitstring) or $m \log n$ (remembering an index takes $\log n$ space).
\end{example}

\begin{claim}
Any deterministic algorithm for $F_0$ requires $\Omega(\min(n,m))$ bits of space ~\cite{Alon}.
\begin{proof}
We will prove this via an \textit{encoding argument}.
Suppose we had a space $S$ streaming algorithm, we will show how to then design a compression mapping $n$-bit string to $S$-bit strings.

Alice receives $\{0,1\}^n$ and wants to compress it using $F_0$ algorithm $A$. Alice creates and artifical stream and uses the memory content of $A$ as compressions.

\underline{Encode(x)}: Create stream containing $\{i | x_i = 1\}$ runs $A$ and outputs memory contents of $A$.

\underline{Decode(x)}: Given memory contents of $A$, we will continue running $A$ and assign the $x_i$ as follows:

\begin{verbatim}
z = query(A)
for i = 1:n
    insert i into stream
    if query(A) == z
        x_i = 1
    else
        x_i = 0
    end
    z = query(Z)
end
\end{verbatim}
\end{proof}
\end{claim}

So we can't do Deterministic/exact $F_0$. So what about:
\begin{itemize}
\item Deterministic/exact: IMPOSSIBLE
\item Det./approx: IMPOSSIBLE
\item Rand/exact: IMPOSSIBLE
\item Rand/approx: POSSIBLE ~\cite{FlajoletM85}
\end{itemize}

Intuition for why we can beat linear space:

\underline{Goal:} Develop a randomized algorithm $\mathcal{A}$ s.t for all streams $S$, $\P(| A(s) - F_0 |) > \epsilon \cdot F_0) < \frac{1}{3}$.

We will develop an \textbf{idealized} randomized algorithm ~\cite{FlajoletM85}:

Suppose we are given a totally random uniform hash function $h: [n] \to [0,1] \subset \R$. In practice we can get away with a reasonable hash function instead that does not take this much space.

Algorithm:

Let $z = \min_{i \in S} h(i)$. To answer query, output $\frac{1}{z} - 1$.

To improve the algorithm, we can pick $h_1, \ldots, h_R$ independent has functions where $R = \Theta(\frac{1}{\epsilon^2})$.
Now, let $z_j = \min_{i \in S} h_j(i)$ and output $\frac{1}{\bar{z}} - 1$ where $\bar{z} = \frac{1}{R} \sum_{j=1}^R z_j$.

Let's look at $h$. We're hashing $F_0$ items into $[0,1]$, so we expect them to be roughly evenly spaced with gap $\frac{1}{F_0 + 1}$, with 
\[
\E z = \frac{1}{F_0 + 1}
\]
or more formally
\[
\E z = \int_0^1 \P(z > x) \dif x = \int_0^1 \P(\forall i \in S, h(i) > X) = \int_0^1 (1-x)F_0 \dif x
\]

\[
\E z^2 = \frac{2}{(F_0 + 1)(F_0 + 2)}
\]\[
\P(| z - \E z | > \epsilon \cdot \E z) - \frac{\Var[z]}{\epsilon^2 \cdot (\E z)^2} \leq \frac{\E z^2}{\epsilon^2 \cdot (\E z)^2}
\]

\paragraph{Bibliography.}

\bibliographystyle{alpha}

\begin{thebibliography}{{Hus}11}

\bibitem[AMS99]{Alon}
Noga Alon, Yossi Matias, and Mario Szegedy.
\newblock The space complexity of approximating the frequency moments.
\newblock {\em J. Comput. Syst. Sci.}, 58(1):137--147, February 1999.

\bibitem[BH06]{BjorklundH06}
Andreas Bj{\"{o}}rklund and Thore Husfeldt.
\newblock Inclusion--exclusion algorithms for counting set partitions.
\newblock In {\em 47th Annual {IEEE} Symposium on Foundations of Computer
  Science {(FOCS} 2006), 21-24 October 2006, Berkeley, California, USA,
  Proceedings}, pages 575--582. {IEEE} Computer Society, 2006.

\bibitem[FM85]{FlajoletM85}
Philippe Flajolet and G.~Nigel Martin.
\newblock Probabilistic counting algorithms for data base applications.
\newblock pages 182--209, 1985.

\bibitem[{Hus}11]{2011arXiv1105.2942H}
T.~{Husfeldt}.
\newblock {Invitation to Algorithmic Uses of Inclusion-Exclusion}.
\newblock {\em ArXiv e-prints}, May 2011.

\bibitem[Koi06]{koivisto20062}
Mikko Koivisto.
\newblock An $\ohstar(2^n)$ algorithm for graph coloring and other partitioning
  problems via inclusion--exclusion.
\newblock In {\em Foundations of Computer Science, 2006. FOCS'06. 47th Annual
  IEEE Symposium on}, pages 583--590. IEEE, 2006.

\bibitem[Mut05]{muthukrishnan2005data}
S~Muthukrishnan.
\newblock {\em Data streams: Algorithms and applications}.
\newblock Now Publishers Inc, 2005.

\end{thebibliography}

\end{document}
