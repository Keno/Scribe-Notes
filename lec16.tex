\documentclass[11pt]{article}
\usepackage[latin9]{inputenc}
\usepackage{amsmath}
\usepackage{amssymb}

\makeatletter
%%%%%%%%%%%%%%%%%%%%%%%%%%%%%% User specified LaTeX commands.

\usepackage{amsthm}

\DeclareMathOperator*{\E}{\mathbb{E}}
\let\Pr\relax
\DeclareMathOperator*{\Pr}{\mathbb{P}}

\newcommand{\eps}{\varepsilon}
\newcommand{\inprod}[1]{\left\langle #1 \right\rangle}
\newcommand{\R}{\mathbb{R}}

\newcommand{\handout}[5]{
  \noindent
  \begin{center}
  \framebox{
    \vbox{
      \hbox to 5.78in { {\bf CS 224: Advanced Algorithms } \hfill #2 }
      \vspace{4mm}
      \hbox to 5.78in { {\Large \hfill #5  \hfill} }
      \vspace{2mm}
      \hbox to 5.78in { {\em #3 \hfill #4} }
    }
  }
  \end{center}
  \vspace*{4mm}
}

\newcommand{\lecture}[4]{\handout{#1}{#2}{#3}{Scribe: #4}{Lecture #1}}

\newtheorem{theorem}{Theorem}
\newtheorem{corollary}[theorem]{Corollary}
\newtheorem{lemma}[theorem]{Lemma}
\newtheorem{observation}[theorem]{Observation}
\newtheorem{proposition}[theorem]{Proposition}
\newtheorem{definition}[theorem]{Definition}
\newtheorem{claim}[theorem]{Claim}
\newtheorem{fact}[theorem]{Fact}
\newtheorem{assumption}[theorem]{Assumption}

% 1-inch margins, from fullpage.sty by H.Partl, Version 2, Dec. 15, 1988.
\topmargin 0pt
\advance \topmargin by -\headheight
\advance \topmargin by -\headsep
\textheight 8.9in
\oddsidemargin 0pt
\evensidemargin \oddsidemargin
\marginparwidth 0.5in
\textwidth 6.5in

\parindent 0in
\parskip 1.5ex

\makeatother

\begin{document}
\lecture{16 --- October 23, 2014}{Fall 2014}{Prof.\ Jelani Nelson}{Colin
Lu}


\section{Overview}

In the last lecture we explored the simplex algorithm for solving
linear programs. While it runs quickly for many practical applications,
it actually does not run in polynomial time.

In this lecture we will finish analyzing the simplex algorithm, and
then look into strong duality and complementary slackness. We will
then overview a polynomial time algorithm for solving linear programs,
the ellipsoid algorithm. We will also introduce the Path-following
interior point algoirhtm for solving linear programs, which is polynomial
time, and faster in practice than the ellipsoid algorithm.


\section{Simplex stuff}

As before, we wish to use the simplex algorithm to solve the linear
program: $\min c^{T}x|Ax=B,x\geq0$.

The simplex algorithm moves from vertex to vertex of the polytope
$P=\{x:Ax-b,x\geq0\}$. A vertex $x\in P$ is a point such that $A_{x}$
has full column rank, where $A_{x}$ is a submatrix of $A$ specified
by indices $B_{x}'=\{j:x_{j}>0\}$. Since the row rank is at most
$m$, we have that $|B_{x}'|\leq m$.

For each matrix $x$, assign basis $B_{x}\subset\{1,...,n\}$ with
$|B_{x}|=m$ such that $A_{B_{x}}$ has rank exactly $m$. We can
think of $B_{x}$ as $B'_{x}$ with additional columns so that it
has full rank.

In the simplex algorithm, we start at some vertex $x^{(0)}$ (we explored
the process of selecting this vertex in the previous lecture), and
greedily move to subsequent vertices until some halting condition.

At any time, we represent our current vertex by some basis $B\subset\{1,...,n\}$,
$|B|=m$, and let $N=[n]\backslash B$.

We can think of the linear program as $\min c_{B}^{T}x_{B}+C_{N}^{T}x_{N}$
subject to $A_{B}x_{B}+A_{N}x_{N}=b$, with $x_{B},x_{N}\geq0$.

By muliplying both sides, we get that this is the same as minimizing
$c_{B}^{T}A_{B}^{-1}b+(c_{N}-A_{N}^{T}(A_{B}^{-1})^{T}c_{B})^{T}x_{N}$.

We define $\tilde{c}_{N}$ as $(c_{N}-A_{N}^{T}(A_{B}^{-1})^{T}c_{B})^{T}$,
the ``reduced cost''

The simplex algorithm proceeds as follows:
\begin{enumerate}
\item At first pass, assume $v_{B}>0$. ($v$ is the current vertex).
\item While $\exists j\in N$ such that $\tilde{c}_{j}<0$, we increase
the $j$th coordinate. This forces a decrease in some coordinates
in $v$, according to the equation $v_{B}=A_{B}^{-1}(b-A_{N}v_{N})$.
We decrease the $j$th coordinate until some coordinate of $v$ is
0. 
\item Otherwise, if $\tilde{c}_{j}\ge0$ for all $j$, then halt and declare
that our current solution is optimal.
\end{enumerate}

\subsection{Pivoting Rules}

However, in reality, we can't just assume that $v_{B}>0$. If $v_{B}>0$
doesn't hold, and there exists some index $j\in N$ such that $\tilde{c}_{j}<0$,
we will add $j$ to $B$ and kick out $k$ from $B$. However, there
will often be a lot of choices for indices $j$ to add, and indices
$k$ to remove. 

Multiple ``pivoting'' rules exist for choosing $j$ and $k$ exist
that provably avoid infinite loops, although some rules that can result
in infinite loops are used in practice occasionally.

One such rule that provably terminates is Bland's rule: 
\begin{enumerate}
\item Pick $j$ to minimize $\tilde{c}_{j}$. If multiple such $j$ exist,
then just take the smallest such index $j$. 
\item Define $r\in R^{n}$, $r_{B}=0$, $r_{N}=\tilde{c}_{N}$. $A'=A_{B}^{-1}A$,
$b'=A_{B}^{-1}b$, and $q=\min_{k}\{\frac{b'_{k}}{A'_{k,j}}|A'_{k.j}>0\}$
\item Pick $k$ to minimize $q$. If there is a tie, pick the smallest index
$k$ minimizing $q$.
\end{enumerate}
Bland's rule can be shown to never result in an infinite loop. See
the original paper \cite{BlandMOR77} for a proof, or the lecture notes \cite{Ye}.


\subsection{Run-time}

We will now analyze the running time of the simplex algorithm. At
each iteration of the simplex algorithm, we take polynomial time to
decrease the $j$th coordinate to perform a pivot. 

The number of iterations is bounded above by the number of vertices,
which is at most $\binom{n}{m}$ (since we can specify a vertex by
its basis elements). In fact, it can be shown that the number of vertices
is at most $\binom{n-\frac{m}{2}}{\frac{m}{2}}$. This follows from
the ``Upper Bound Theorem'' proven by McMullen in\cite{McMullen71}.

If we wanted to show a better bound than simply $\#vertices\times iterationtime$,
one necessary step would be to show that the maximum diameter (longest
shortest path) of a graph corresponding to a polytope in $d$ dimensions
defined by $m$ hyperplanes is polynomial in $m$ and $d$ . If we
were unable to do this, then in the worst case we might simply start
``too far'' away from the optimal vertex.

In 1957, Warren Hirsch conjectured that $diam\leq m-d$. His conjecture
remained open for more than 50 years, before Santos disproved it in
\cite{Santos11} with an explicit construction in which $d=43$, $m=86$,
$diam\geq44$. He also showed that for fixed $d$, $\epsilon>0$,
there exists an infinite family of polytopes with $m\rightarrow\infty$
that have diameter at least $(1+\epsilon)m$.

However, even if we managed to show that the maximum diameter of a
polytope graph is small, we would also need to show that the simplex
algorithm actually finds short paths within such graphs. It's still
an open question as to whether there is a fast (i.e. worst case polynomial
time) implementation of the simplex algorithm.


\subsubsection{Simplex's speed in practice}

Spielman and Teng in \cite{SpielmanT2014} used ``smoothed analysis''
to explore why the simplex algorithms we use are fast in practice,
despite the fact that haven't shown to be polynomial time in general.

In smoothed analysis, we consider the average running time of a worst
case instance with some small random independent noise added to all
entries. Formally, the running time of an algorithm $A$ in smoothed
analysis is $\min_{x}\mathbb{E}(runtime(A(\tilde{x})))$ , where $\tilde{x}$
is $x$ with random noise added, $x$ is some input to the algorithm,
and the expectation is taken over the noise added to $x$.

Spielman and Teng showed that simplex algorithms have smoothed analysis
times that are polynomial.


\section{Strong Duality}

Recall that the dual of our linear program $\min c^{T}x|Ax=B,x\geq0$,
is the program $\max y^{T}b|A^{T}y\leq c$. Note that when we have
equality constraints $Ax=b$, the dual variable $y$ does not need
to be non-negative.

There are multiple ways to get strong duality. We will show it as
a corollary of the simplex algorithm.

Define the ``shadow price vector'', $y=(A_{B}^{-1})^{T}c_{B}$.

\begin{claim}
  $c^{T}v_{B}=b^{T}y$ .
\end{claim}

\begin{proof}
  Recall that we have $A_{B}v_{B}=b$, which tells us that $v_{B}=A_{B}^{-1}b$.
  Hence, $c^{T}v_{B}=c^{T}A_{B}^{-1}b=b^{T}(A_{B}^{-1})^{T}c_{B}=b^{T}y$.
\end{proof}

This does not quite give us strong duality: we need to also show that
$y$ is a feasible solution to the dual.

Define the dual slack as $r=c-A^{T}y$. $y$ being feasible is equivalent
to $r\geq0$. The definition of $y$ gives us that $r_{B}=0$, and
that $r_{N}=\tilde{c}_{N}$. Note that the simplex algorithm halts
when $\tilde{c}_{N}\geq0$, which implies that at this point, we have
dual feasibility.


\section{Complementary Slackness}

We can define the ``complementarity gap'' as $x^{T}r=x^{T}(c-A^{T}y)=c^{T}x-(Ax)^{T}y=c^{T}x-b^{T}y$.


\begin{theorem}
  (complementary slackness): If $x^{*}$ and $y^{*}$ are optimal
  for the primal and dual respectively, then for all $i$, either $x_{i}=0$,
  or $r_{i}=(c-A^{T}y)_{i}=0$.
\end{theorem}

\begin{proof}
  At optimality, by strong duality, there is no gap between the
  primal and dual solutions, so $c^{T}x-b^{T}y=0=\sum x_{i}r_{i}$.
  Since each $x_{i},r_{i}$ is non-negative, this can only be true if
  for each $i$, either $x_{i}=0$ or $r_{i}=0$, as required.
\end{proof}

\section{Ellipsoid Algorithm}

The first polynomial-time algorithm for solving linear programs is
the ellipsoid algorithm, which was introduced by Shor, and applied
to linear programs by Khachian in \cite{Khachian79}. We won't cover the full details of ellipsoid here but will just
give a flavor of what goes on. If you want to see more details, see for example the notes from lectures 4 through 6 of \cite{Goemans01}.

The ellipsoid algorithm itself tests the feasibility of some polytope
$P=\{x:Ax\le b\}$.

The ellipsoid algorithm only uses the constraints in one way: it wants
an oracle that will tell it, given an infeasible point, a constraint
that it violates. However, if an oracle is given indepednent of the
contraints themselves, then that can be used instead of directly referencing
them.

Given some $a\in\R^n,A\in\R^{m\times n},$ the ellipsoid $E(a,A)=\{x|(x-a)^{T}A^{-1}(x-a)\leq1\}$.

The ellipsoid algorithm is used to test the feasibility of some polytope
$P=\{x:Ax\leq b\}$.

The basic idea of the algorithm is to: 
\begin{itemize}
\item Start with some ellipsoid $E_{0}$ which is guaranteed to contain
$P$. One way to do this is to compute the bit complexity of $P$,
and use this to put an upper bound on the radius of a sphere needed
to contain it.
\item Let $a$ be the center of the current ellipsoid $E$. While $a$ is
not in $P$, there exists some hyperplane $H$ corresponding to some
constraint, such that $P$ lies entirely on the opposite side of $H$
from $a$. Create a new ellipsoid $E'$ which still contains $P$
and has smaller volume and repeat. Khachian showed that there is a
way to do this such that $\frac{\mbox{vol}(E')}{\mbox{vol}(E)}\le e^{\frac{-1}{2n}}$.
\item If $a\in P$, return $a$. If $\mbox{vol}(E)$ has become ``too small'',
declare that $P$ is empty.
\end{itemize}
This has a problem, in that the polytope $P$ could lie in a lower
dimensional subspace, such that it has zero volume, but still contains
points. To fix this, we do some pre-processing:

Theorem: Let $L$ be the ``bit complexity'' of $P$. $L\sim\log(\max_{i,j}|A_{ij}|+\|b\|_{\infty}+m+n)$.
Define $P'=\{x:Ax\leq b+2^{-L},\forall j,-2^{L}\leq x_{j}\leq2^{L}\}$.
Then $P\mbox{ is infeasible \ensuremath{\iff}\ensuremath{P'}is infeasible}$.
Thus, we can then apply the above algorithm to $P'$ in order 


\subsection{Application to solving linear program}

Suppose we actually want to solve a linear program $\min c^{T}x$,
such that $Ax=b$, $x\geq0$.

We can solve it using the ellipsoid algoirthm as follows:
\begin{enumerate}
\item Check if $P=\{x|Ax=b\}$ is feasible. If not, output ``not feasible''.
\item Write the dual: $\max b^{T}y$ such that $A^{T}y\leq c$. Check that
$P_{dual}=\{y:A^{T}y\leq c\}$ is feasible. If not, then output ``unbounded''.
\item Find $(x,y)$ that is feasible for the polytope defined by $\{Ax=b,x\geq0,A^{T}y\leq c.c^{T}x=b^{T}y\}$.
This pair $(x,y)$ give optimal solutions for the primal and dual
respectively.
\end{enumerate}

\subsubsection{Run-time of ellipsoid algorithm on linear programs}

For the $k$th ellipsoid $E_{k}$, we have that $\mbox{vol}(E_{k})\leq e^{\frac{-k}{2n}}\mbox{vol}(E_{0})$.
We stop the algorithm when this volume is less than that of $P'$,
which takes a polynomial number of iterations.


\section{Path-following interior point}

Path-following interior point is another algorithm that gives polynomial
time solutions to linear programs. It was introduced by Karmarkar
in \cite{Karmarkar84}.

The interior point algorithm gradually tries to reach an optimal vertex
while staying in the interior of $P$. It introduces a variable that
makes it very easy to satisfy the constraints, but is penalized heavily
in the objective function. With this variable, it is very easy to
find a starting interior point, and then iterates on this point.

For the interior point algorithm, we'll look at linear programs of
the form $\min c^{T}x|Ax\geq b$, and define the slackness $S(x)=Ax-b$.
For interior point, we attempt to compute $\min_{x\in\mathbb{R}^{n}}\lambda c^{T}x+p(S(x))$
where $p$ is a ``barrier function'' such that $p(z)\rightarrow\infty$
if $z_{i}\rightarrow0$ for any $i$. In particular, we'll use the
``log barrier'' function, $p(z)=-\sum_{i=1}^{m}\ln(z_{i})$.

For very small $\lambda$, $\min_{x\in\mathbb{R}^{n}}\lambda c^{T}x+p(S(x))$
is a kinda of central point in the polytope, and we iterate, incrementing
$\lambda$ and modifying $x$, until $\min_{x\in\mathbb{R}^{n}}\lambda c^{T}x+p(S(x))$
is approximately a vertex, and can then be ``rounded'' to the nearest
vertex. We'll analyze the interior point algorithm in more detail
in the next lecture.
\begin{thebibliography}{1}
\bibitem{BlandMOR77}
Robert~Bland.
\newblock New Finite Pivoting Rules for the Simplex Method.
\newblock {\em Mathematics of Operations Research}, 2(2):103--107, 1977.

\bibitem{McMullen71}Peter McMullen. The maximum numbers of faces
of a a convex polytope. Mathematika 17:179-184, 1971.

\bibitem{Santos11}Francisco Santos. A counterexample to the Hirsch
conjecture, Annals of Mathematics 176 (10: 383-412, doi:10.4007/annals.2012.176.1.7,
2011.

\bibitem{SpielmanT2014}Daniel A. Spielman, Shang-Hua Teng. Smoothed
analysis of algorithms. Why the simplex algorithm usually takes polynomial
time. J. ACM 51(3): 385-463, 2004.

\bibitem{Goemans01}
Michel~Goemans.
\newblock Combinatorial Optimization Lecture notes on the ellipsoid algorithm.
\texttt{http://www-math.mit.edu/{\textasciitilde}goemans/18433S09/ellipsoid.pdf}.

\bibitem{Khachian79}Leonid Khachian. A polynomial algorithm in linear
programming. Doklady Akademi Nauk USSR 244:1093-1096, 1979.

\bibitem{Karmarkar84}Narendra Karmarkar. A New Polynomial Time Algorithm
for Linear Programming. Combinatorica, Vol 4, 4:373-395, 1984.

\bibitem{Ye}
Yinyu Ye.
\newblock MS{\&}E310 Linear Optimization Autumn Course Notes on Bland's Rule.
\newblock \texttt{http://web.stanford.edu/class/msande310/blandrule.pdf}.
\end{thebibliography}

\end{document}
